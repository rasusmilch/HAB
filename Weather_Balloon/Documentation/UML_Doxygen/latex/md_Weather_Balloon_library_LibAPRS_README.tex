Lib\+A\+P\+RS is an Arduino I\+DE library that makes it easy to send and receive A\+P\+RS packets with a \href{http://unsigned.io/micromodem}{\tt Micro\+Modem}-\/compatible modem.

You can buy a complete modem from \href{http://unsigned.io/shop}{\tt my shop}, or you can build one yourself pretty easily. Take a look at the documentation in the \href{https://github.com/markqvist/MicroModem}{\tt Micro\+Modem} repository for information and getting started guides!

See the example included in the library for info on how to use it!

\subsection*{Some features}


\begin{DoxyItemize}
\item Send and receive A\+X.\+25 A\+P\+RS packets
\item Full modulation and demodulation in software
\item Easy configuration of callsign and path settings
\item Easily process incoming packets
\item Shorthand functions for sending location updates and messages, so you don\textquotesingle{}t need to manually create the packets
\item Ability to send raw packets
\item Support for settings A\+P\+RS symbols
\item Support for power/height/gain info in location updates
\item Can run with open squelch
\end{DoxyItemize}

\subsection*{Installation}

Place the \char`\"{}\+Lib\+A\+P\+R\+S\char`\"{} folder (the one in the same folder as this readme file) inside your Arduino \char`\"{}libraries\char`\"{} folder. That\textquotesingle{}s all!

\subsection*{Getting started}

You should read through the \char`\"{}\+Basic\+\_\+usage\char`\"{} example included with the library. It contains an explanation of all the functions and a basic sketch to get you up and running with sending and receiving packets.

\subsection*{Got bugs?}

This library is very early, and being actively developed all the time. This means you should expect to find bugs. If you do, please report them here, so I can fix them! It also means I might have to make changes that will break code, and that you will have to rewrite your sketch. If you don\textquotesingle{}t think that sounds good, wait a little while for a stable release of the library \+:) 
\hyperlink{_firmata_8h_a986c68fac4f21302213e609c277710d7}{Firmata}

\href{https://gitter.im/firmata/arduino?utm_source=badge&utm_medium=badge&utm_campaign=pr-badge&utm_content=badge}{\tt }

Firmata is a protocol for communicating with microcontrollers from software on a host computer. The \href{https://github.com/firmata/protocol}{\tt protocol} can be implemented in firmware on any microcontroller architecture as well as software on any host computer software package. The Arduino repository described here is a Firmata library for Arduino and Arduino-\/compatible devices. If you would like to contribute to Firmata, please see the \href{#contributing}{\tt Contributing} section below.

\subsection*{Usage}

There are two main models of usage of Firmata. In one model, the author of the Arduino sketch uses the various methods provided by the Firmata library to selectively send and receive data between the Arduino device and the software running on the host computer. For example, a user can send analog data to the host using {\ttfamily Firmata.\+send\+Analog(analog\+Pin, analog\+Read(analog\+Pin))} or send data packed in a string using {\ttfamily Firmata.\+send\+String(string\+To\+Send)}. See File -\/$>$ Examples -\/$>$ Firmata -\/$>$ Analog\+Firmata \& Echo\+String respectively for examples.

The second and more common model is to load a general purpose sketch called Standard\+Firmata (or one of the variants such as Standard\+Firmata\+Plus or Standard\+Firmata\+Ethernet depending on your needs) on the Arduino board and then use the host computer exclusively to interact with the Arduino board. Standard\+Firmata is located in the Arduino I\+DE in File -\/$>$ Examples -\/$>$ Firmata.

\subsection*{Firmata Client Libraries}

Most of the time you will be interacting with Arduino with a client library on the host computers. Several Firmata client libraries have been implemented in a variety of popular programming languages\+:


\begin{DoxyItemize}
\item processing
\begin{DoxyItemize}
\item \mbox{[}\href{https://github.com/firmata/processing}{\tt https\+://github.\+com/firmata/processing}\mbox{]}
\item \mbox{[}\href{http://funnel.cc}{\tt http\+://funnel.\+cc}\mbox{]}
\end{DoxyItemize}
\item python
\begin{DoxyItemize}
\item \mbox{[}\href{https://github.com/MrYsLab/pymata-aio}{\tt https\+://github.\+com/\+Mr\+Ys\+Lab/pymata-\/aio}\mbox{]}
\item \mbox{[}\href{https://github.com/MrYsLab/PyMata}{\tt https\+://github.\+com/\+Mr\+Ys\+Lab/\+Py\+Mata}\mbox{]}
\item \mbox{[}\href{https://github.com/tino/pyFirmata}{\tt https\+://github.\+com/tino/py\+Firmata}\mbox{]}
\item \mbox{[}\href{https://github.com/lupeke/python-firmata}{\tt https\+://github.\+com/lupeke/python-\/firmata}\mbox{]}
\item \mbox{[}\href{https://github.com/firmata/pyduino}{\tt https\+://github.\+com/firmata/pyduino}\mbox{]}
\end{DoxyItemize}
\item perl
\begin{DoxyItemize}
\item \mbox{[}\href{https://github.com/ntruchsess/perl-firmata}{\tt https\+://github.\+com/ntruchsess/perl-\/firmata}\mbox{]}
\item \mbox{[}\href{https://github.com/rcaputo/rx-firmata}{\tt https\+://github.\+com/rcaputo/rx-\/firmata}\mbox{]}
\end{DoxyItemize}
\item ruby
\begin{DoxyItemize}
\item \mbox{[}\href{https://github.com/hardbap/firmata}{\tt https\+://github.\+com/hardbap/firmata}\mbox{]}
\item \mbox{[}\href{https://github.com/PlasticLizard/rufinol}{\tt https\+://github.\+com/\+Plastic\+Lizard/rufinol}\mbox{]}
\item \mbox{[}\href{http://funnel.cc}{\tt http\+://funnel.\+cc}\mbox{]}
\end{DoxyItemize}
\item clojure
\begin{DoxyItemize}
\item \mbox{[}\href{https://github.com/nakkaya/clodiuno}{\tt https\+://github.\+com/nakkaya/clodiuno}\mbox{]}
\item \mbox{[}\href{https://github.com/peterschwarz/clj-firmata}{\tt https\+://github.\+com/peterschwarz/clj-\/firmata}\mbox{]}
\end{DoxyItemize}
\item javascript
\begin{DoxyItemize}
\item \mbox{[}\href{https://github.com/jgautier/firmata}{\tt https\+://github.\+com/jgautier/firmata}\mbox{]}
\item \mbox{[}\href{https://github.com/rwldrn/johnny-five}{\tt https\+://github.\+com/rwldrn/johnny-\/five}\mbox{]}
\item \mbox{[}\href{http://breakoutjs.com}{\tt http\+://breakoutjs.\+com}\mbox{]}
\end{DoxyItemize}
\item java
\begin{DoxyItemize}
\item \mbox{[}\href{https://github.com/kurbatov/firmata4j}{\tt https\+://github.\+com/kurbatov/firmata4j}\mbox{]}
\item \mbox{[}\href{https://github.com/4ntoine/Firmata}{\tt https\+://github.\+com/4ntoine/\+Firmata}\mbox{]}
\item \mbox{[}\href{https://github.com/reapzor/FiloFirmata}{\tt https\+://github.\+com/reapzor/\+Filo\+Firmata}\mbox{]}
\end{DoxyItemize}
\item .N\+ET
\begin{DoxyItemize}
\item \mbox{[}\href{https://github.com/SolidSoils/Arduino}{\tt https\+://github.\+com/\+Solid\+Soils/\+Arduino}\mbox{]}
\item \mbox{[}\href{http://www.imagitronics.org/projects/firmatanet/}{\tt http\+://www.\+imagitronics.\+org/projects/firmatanet/}\mbox{]}
\end{DoxyItemize}
\item Flash/\+A\+S3
\begin{DoxyItemize}
\item \mbox{[}\href{http://funnel.cc}{\tt http\+://funnel.\+cc}\mbox{]}
\item \mbox{[}\href{http://code.google.com/p/as3glue/}{\tt http\+://code.\+google.\+com/p/as3glue/}\mbox{]}
\end{DoxyItemize}
\item P\+HP
\begin{DoxyItemize}
\item \mbox{[}\href{https://github.com/ThomasWeinert/carica-firmata}{\tt https\+://github.\+com/\+Thomas\+Weinert/carica-\/firmata}\mbox{]}
\item \mbox{[}\href{https://github.com/oasynnoum/phpmake_firmata}{\tt https\+://github.\+com/oasynnoum/phpmake\+\_\+firmata}\mbox{]}
\end{DoxyItemize}
\item Haskell
\begin{DoxyItemize}
\item \mbox{[}\href{http://hackage.haskell.org/package/hArduino}{\tt http\+://hackage.\+haskell.\+org/package/h\+Arduino}\mbox{]}
\end{DoxyItemize}
\item i\+OS
\begin{DoxyItemize}
\item \mbox{[}\href{https://github.com/jacobrosenthal/iosfirmata}{\tt https\+://github.\+com/jacobrosenthal/iosfirmata}\mbox{]}
\end{DoxyItemize}
\item Dart
\begin{DoxyItemize}
\item \mbox{[}\href{https://github.com/nfrancois/firmata}{\tt https\+://github.\+com/nfrancois/firmata}\mbox{]}
\end{DoxyItemize}
\item Max/\+M\+SP
\begin{DoxyItemize}
\item \mbox{[}\href{http://www.maxuino.org/}{\tt http\+://www.\+maxuino.\+org/}\mbox{]}
\end{DoxyItemize}
\item Elixir
\begin{DoxyItemize}
\item \mbox{[}\href{https://github.com/kfatehi/firmata}{\tt https\+://github.\+com/kfatehi/firmata}\mbox{]}
\end{DoxyItemize}
\item Modelica
\begin{DoxyItemize}
\item \mbox{[}\href{https://www.wolfram.com/system-modeler/libraries/model-plug/}{\tt https\+://www.\+wolfram.\+com/system-\/modeler/libraries/model-\/plug/}\mbox{]}
\end{DoxyItemize}
\item Go
\begin{DoxyItemize}
\item \mbox{[}\href{https://github.com/kraman/go-firmata}{\tt https\+://github.\+com/kraman/go-\/firmata}\mbox{]}
\end{DoxyItemize}
\item vvvv
\begin{DoxyItemize}
\item \mbox{[}\href{https://vvvv.org/blog/arduino-second-service}{\tt https\+://vvvv.\+org/blog/arduino-\/second-\/service}\mbox{]}
\end{DoxyItemize}
\item open\+Frameworks
\begin{DoxyItemize}
\item \mbox{[}\href{http://openframeworks.cc/documentation/communication/ofArduino/}{\tt http\+://openframeworks.\+cc/documentation/communication/of\+Arduino/}\mbox{]}
\end{DoxyItemize}
\item Rust
\begin{DoxyItemize}
\item \mbox{[}\href{https://github.com/zankich/rust-firmata}{\tt https\+://github.\+com/zankich/rust-\/firmata}\mbox{]}
\end{DoxyItemize}
\end{DoxyItemize}

Note\+: The above libraries may support various versions of the Firmata protocol and therefore may not support all features of the latest Firmata spec nor all Arduino and Arduino-\/compatible boards. Refer to the respective projects for details.

\subsection*{Updating Firmata in the Arduino I\+DE -\/ Arduino 1.\+6.\+4 and higher}

If you want to update to the latest stable version\+:


\begin{DoxyEnumerate}
\item Open the Arduino I\+DE and navigate to\+: {\ttfamily Sketch $>$ Include Library $>$ Manage Libraries}
\item Filter by \char`\"{}\+Firmata\char`\"{} and click on the \char`\"{}\+Firmata by Firmata Developers\char`\"{} item in the list of results.
\item Click the {\ttfamily Select version} dropdown and select the most recent version (note you can also install previous versions)
\item Click {\ttfamily Install}.
\end{DoxyEnumerate}

\subsubsection*{Cloning Firmata}

If you are contributing to Firmata or otherwise need a version newer than the latest tagged release, you can clone Firmata directly to your Arduino/libraries/ directory (where 3rd party libraries are installed). This only works for Arduino 1.\+6.\+4 and higher, for older versions you need to clone into the Arduino application directory (see section below titled \char`\"{}\+Using the Source code rather than release archive\char`\"{}). Be sure to change the name to Firmata as follows\+:


\begin{DoxyCode}
1 $ git clone git@github.com:firmata/arduino.git ~/Documents/Arduino/libraries/Firmata
\end{DoxyCode}


{\itshape Update path above if you\textquotesingle{}re using Windows or Linux or changed the default Arduino directory on OS X}

\subsection*{Updating Firmata in the Arduino I\+DE -\/ older versions ($<$= 1.\+6.\+3 or 1.\+0.\+x)}

Download the latest \href{https://github.com/firmata/arduino/releases/tag/2.5.4}{\tt release} (for Arduino 1.\+0.\+x or Arduino 1.\+5.\+6 or higher) and replace the existing Firmata folder in your Arduino application. See the instructions below for your platform.

{\itshape Note that Arduino 1.\+5.\+0 -\/ 1.\+5.\+5 are not supported. Please use Arduino 1.\+5.\+6 or higher (or Arduino 1.\+0.\+5 or 1.\+0.\+6).}

\subsubsection*{Mac O\+SX\+:}

The Firmata library is contained within the Arduino package.


\begin{DoxyEnumerate}
\item Navigate to the Arduino application
\item Right click on the application icon and select {\ttfamily Show Package Contents}
\item Navigate to\+: {\ttfamily /\+Contents/\+Resources/\+Java/libraries/} and replace the existing {\ttfamily Firmata} folder with latest \href{https://github.com/firmata/arduino/releases/tag/2.5.4}{\tt Firmata release} (note there is a different download for Arduino 1.\+0.\+x vs 1.\+6.\+x)
\item Restart the Arduino application and the latest version of Firmata will be available.
\end{DoxyEnumerate}

{\itshape If you are using the Java 7 version of Arduino 1.\+5.\+7 or higher, the file path will differ slightly\+: {\ttfamily Contents/\+Java/libraries/\+Firmata} (no Resources directory).}

\subsubsection*{Windows\+:}


\begin{DoxyEnumerate}
\item Navigate to {\ttfamily c\+:/\+Program\textbackslash{} Files/arduino-\/1.\+x/libraries/} and replace the existing {\ttfamily Firmata} folder with the latest \href{https://github.com/firmata/arduino/releases/tag/2.5.4}{\tt Firmata release} (note there is a different download for Arduino 1.\+0.\+x vs 1.\+6.\+x).
\item Restart the Arduino application and the latest version of Firmata will be available.
\end{DoxyEnumerate}

{\itshape Update the path and Arduino version as necessary}

\subsubsection*{Linux\+:}


\begin{DoxyEnumerate}
\item Navigate to {\ttfamily $\sim$/arduino-\/1.x/libraries/} and replace the existing {\ttfamily Firmata} folder with the latest \href{https://github.com/firmata/arduino/releases/tag/2.5.4}{\tt Firmata release} (note there is a different download for Arduino 1.\+0.\+x vs 1.\+6.\+x).
\item Restart the Arduino application and the latest version of Firmata will be available.
\end{DoxyEnumerate}

{\itshape Update the path and Arduino version as necessary}

\subsubsection*{Using the Source code rather than release archive (only for versions older than Arduino 1.\+6.\+3)}

{\itshape It is recommended you update to Arduino 1.\+6.\+4 or higher if possible, that way you can clone directly into the external Arduino/libraries/ directory which persists between Arduino application updates. Otherwise you will need to move your clone each time you update to a newer version of the Arduino I\+DE.}

If you\textquotesingle{}re stuck with an older version of the I\+DE, then follow these keep reading otherwise jump up to the \char`\"{}\+Cloning Firmata section above\char`\"{}.

Clone this repo directly into the core Arduino application libraries directory. If you are using Arduino 1.\+5.\+x or $<$= 1.\+6.\+3, the repo directory structure will not match the Arduino library format, however it should still compile as long as you are using Arduino 1.\+5.\+7 or higher.

You will first need to remove the existing Firmata library, then clone firmata/arduino into an empty Firmata directory\+:


\begin{DoxyCode}
1 $ rm -r /Applications/Arduino.app/Contents/Resources/Java/libraries/Firmata
2 $ git clone git@github.com:firmata/arduino.git
       /Applications/Arduino.app/Contents/Resources/Java/libraries/Firmata
\end{DoxyCode}


{\itshape Update paths if you\textquotesingle{}re using Windows or Linux}

To generate properly formatted versions of Firmata (for Arduino 1.\+0.\+x and Arduino 1.\+6.\+x), run the {\ttfamily release.\+sh} script.

\label{_contributing}%
 \subsection*{Contributing}

If you discover a bug or would like to propose a new feature, please open a new \href{https://github.com/firmata/arduino/issues?sort=created&state=open}{\tt issue}. Due to the limited memory of standard Arduino boards we cannot add every requested feature to Standard\+Firmata. Requests to add new features to Standard\+Firmata will be evaluated by the Firmata developers. However it is still possible to add new features to other Firmata implementations (Firmata is a protocol whereas Standard\+Firmata is just one of many possible implementations).

To contribute, fork this repository and create a new topic branch for the bug, feature or other existing issue you are addressing. Submit the pull request against the {\itshape master} branch.

If you would like to contribute but don\textquotesingle{}t have a specific bugfix or new feature to contribute, you can take on an existing issue, see issues labeled \char`\"{}pull-\/request-\/encouraged\char`\"{}. Add a comment to the issue to express your intent to begin work and/or to get any additional information about the issue.

You must thoroughly test your contributed code. In your pull request, describe tests performed to ensure that no existing code is broken and that any changes maintain backwards compatibility with the existing api. Test on multiple Arduino board variants if possible. We hope to enable some form of automated (or at least semi-\/automated) testing in the future, but for now any tests will need to be executed manually by the contributor and reviewers.

Use \href{http://astyle.sourceforge.net/}{\tt Artistic Style} (astyle) to format your code. Set the following rules for the astyle formatter\+:


\begin{DoxyCode}
1 style = ""
2 indent-spaces = 2
3 indent-classes = true
4 indent-switches = true
5 indent-cases = true
6 indent-col1-comments = true
7 pad-oper = true
8 pad-header = true
9 keep-one-line-statements = true
\end{DoxyCode}


If you happen to use Sublime Text, \href{https://github.com/timonwong/SublimeAStyleFormatter}{\tt this astyle plugin} is helpful. Set the above rules in the user settings file. 
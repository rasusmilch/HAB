Many small embedded systems exist to collect data from sensors, analyse the data, and either take an appropriate action or send that sensor data to another system for processing.

One of the many challenges of embedded systems design is the fact that parts you used today may be out of production tomorrow, or system requirements may change and you may need to choose a different sensor down the road.

Creating new drivers is a relatively easy task, but integrating them into existing systems is both error prone and time consuming since sensors rarely use the exact same units of measurement.

By reducing all data to a single {\bfseries \hyperlink{_adafruit___sensor_8h_structsensors__event__t}{sensors\+\_\+event\+\_\+t}} \textquotesingle{}type\textquotesingle{} and settling on specific, {\bfseries standardised SI units} for each sensor family the same sensor types return values that are comparable with any other similar sensor. This enables you to switch sensor models with very little impact on the rest of the system, which can help mitigate some of the risks and problems of sensor availability and code reuse.

The unified sensor abstraction layer is also useful for data-\/logging and data-\/transmission since you only have one well-\/known type to log or transmit over the air or wire.

\subsection*{Unified Sensor Drivers}

The following drivers are based on the Adafruit Unified Sensor Driver\+:

{\bfseries Accelerometers}
\begin{DoxyItemize}
\item \href{https://github.com/adafruit/Adafruit_ADXL345}{\tt Adafruit\+\_\+\+A\+D\+X\+L345}
\item \href{https://github.com/adafruit/Adafruit_LSM303DLHC}{\tt Adafruit\+\_\+\+L\+S\+M303\+D\+L\+HC}
\item \href{https://github.com/adafruit/Adafruit_MMA8451_Library}{\tt Adafruit\+\_\+\+M\+M\+A8451\+\_\+\+Library}
\end{DoxyItemize}

{\bfseries Gyroscope}
\begin{DoxyItemize}
\item \href{https://github.com/adafruit/Adafruit_L3GD20_U}{\tt Adafruit\+\_\+\+L3\+G\+D20\+\_\+U}
\end{DoxyItemize}

{\bfseries Light}
\begin{DoxyItemize}
\item \href{https://github.com/adafruit/Adafruit_TSL2561}{\tt Adafruit\+\_\+\+T\+S\+L2561}
\item \href{https://github.com/adafruit/Adafruit_TSL2591_Library}{\tt Adafruit\+\_\+\+T\+S\+L2591\+\_\+\+Library}
\end{DoxyItemize}

{\bfseries Magnetometers}
\begin{DoxyItemize}
\item \href{https://github.com/adafruit/Adafruit_LSM303DLHC}{\tt Adafruit\+\_\+\+L\+S\+M303\+D\+L\+HC}
\item \href{https://github.com/adafruit/Adafruit_HMC5883_Unified}{\tt Adafruit\+\_\+\+H\+M\+C5883\+\_\+\+Unified}
\end{DoxyItemize}

{\bfseries Barometric Pressure}
\begin{DoxyItemize}
\item \href{https://github.com/adafruit/Adafruit_BMP085_Unified}{\tt Adafruit\+\_\+\+B\+M\+P085\+\_\+\+Unified}
\item \href{https://github.com/adafruit/Adafruit_BMP183_Unified_Library}{\tt Adafruit\+\_\+\+B\+M\+P183\+\_\+\+Unified\+\_\+\+Library}
\end{DoxyItemize}

{\bfseries Humidity \& Temperature}
\begin{DoxyItemize}
\item \href{https://github.com/adafruit/Adafruit_DHT_Unified}{\tt Adafruit\+\_\+\+D\+H\+T\+\_\+\+Unified}
\end{DoxyItemize}

{\bfseries Orientation}
\begin{DoxyItemize}
\item \href{https://github.com/adafruit/Adafruit_BNO055}{\tt Adafruit\+\_\+\+B\+N\+O055}
\end{DoxyItemize}

\subsection*{How Does it Work?}

Any driver that supports the Adafruit unified sensor abstraction layer will implement the \hyperlink{class_adafruit___sensor}{Adafruit\+\_\+\+Sensor} base class. There are two main typedefs and one enum defined in \hyperlink{_adafruit___sensor_8h}{Adafruit\+\_\+\+Sensor.\+h} that are used to \textquotesingle{}abstract\textquotesingle{} away the sensor details and values\+:

{\bfseries Sensor Types (sensors\+\_\+type\+\_\+t)}

These pre-\/defined sensor types are used to properly handle the two related typedefs below, and allows us determine what types of units the sensor uses, etc.


\begin{DoxyCode}
1 /** Sensor types */
2 typedef enum
3 \{
4   SENSOR\_TYPE\_ACCELEROMETER         = (1),
5   SENSOR\_TYPE\_MAGNETIC\_FIELD        = (2),
6   SENSOR\_TYPE\_ORIENTATION           = (3),
7   SENSOR\_TYPE\_GYROSCOPE             = (4),
8   SENSOR\_TYPE\_LIGHT                 = (5),
9   SENSOR\_TYPE\_PRESSURE              = (6),
10   SENSOR\_TYPE\_PROXIMITY             = (8),
11   SENSOR\_TYPE\_GRAVITY               = (9),
12   SENSOR\_TYPE\_LINEAR\_ACCELERATION   = (10),
13   SENSOR\_TYPE\_ROTATION\_VECTOR       = (11),
14   SENSOR\_TYPE\_RELATIVE\_HUMIDITY     = (12),
15   SENSOR\_TYPE\_AMBIENT\_TEMPERATURE   = (13),
16   SENSOR\_TYPE\_VOLTAGE               = (15),
17   SENSOR\_TYPE\_CURRENT               = (16),
18   SENSOR\_TYPE\_COLOR                 = (17)
19 \} sensors\_type\_t;
\end{DoxyCode}


{\bfseries Sensor Details (\hyperlink{_adafruit___sensor_8h_structsensor__t}{sensor\+\_\+t})}

This typedef describes the specific capabilities of this sensor, and allows us to know what sensor we are using beneath the abstraction layer.


\begin{DoxyCode}
1 /* Sensor details (40 bytes) */
2 /** struct sensor\_s is used to describe basic information about a specific sensor. */
3 typedef struct
4 \{
5     char     name[12];
6     int32\_t  version;
7     int32\_t  sensor\_id;
8     int32\_t  type;
9     float    max\_value;
10     float    min\_value;
11     float    resolution;
12     int32\_t  min\_delay;
13 \} sensor\_t;
\end{DoxyCode}


The individual fields are intended to be used as follows\+:


\begin{DoxyItemize}
\item {\bfseries name}\+: The sensor name or ID, up to a maximum of twelve characters (ex. \char`\"{}\+M\+P\+L115\+A2\char`\"{})
\item {\bfseries version}\+: The version of the sensor HW and the driver to allow us to differentiate versions of the board or driver
\item {\bfseries sensor\+\_\+id}\+: A unique sensor identifier that is used to differentiate this specific sensor instance from any others that are present on the system or in the sensor network
\item {\bfseries type}\+: The sensor type, based on {\bfseries sensors\+\_\+type\+\_\+t} in sensors.\+h
\item {\bfseries max\+\_\+value}\+: The maximum value that this sensor can return (in the appropriate SI unit)
\item {\bfseries min\+\_\+value}\+: The minimum value that this sensor can return (in the appropriate SI unit)
\item {\bfseries resolution}\+: The smallest difference between two values that this sensor can report (in the appropriate SI unit)
\item {\bfseries min\+\_\+delay}\+: The minimum delay in microseconds between two sensor events, or \textquotesingle{}0\textquotesingle{} if there is no constant sensor rate
\end{DoxyItemize}

{\bfseries Sensor Data/\+Events (\hyperlink{_adafruit___sensor_8h_structsensors__event__t}{sensors\+\_\+event\+\_\+t})}

This typedef is used to return sensor data from any sensor supported by the abstraction layer, using standard SI units and scales.


\begin{DoxyCode}
1 /* Sensor event (36 bytes) */
2 /** struct sensor\_event\_s is used to provide a single sensor event in a common format. */
3 typedef struct
4 \{
5     int32\_t version;
6     int32\_t sensor\_id;
7     int32\_t type;
8     int32\_t reserved0;
9     int32\_t timestamp;
10     union
11     \{
12         float           data[4];
13         sensors\_vec\_t   acceleration;
14         sensors\_vec\_t   magnetic;
15         sensors\_vec\_t   orientation;
16         sensors\_vec\_t   gyro;
17         float           temperature;
18         float           distance;
19         float           light;
20         float           pressure;
21         float           relative\_humidity;
22         float           current;
23         float           voltage;
24         sensors\_color\_t color;
25     \};
26 \} sensors\_event\_t;
\end{DoxyCode}
 It includes the following fields\+:


\begin{DoxyItemize}
\item {\bfseries version}\+: Contain \textquotesingle{}sizeof(sensors\+\_\+event\+\_\+t)\textquotesingle{} to identify which version of the A\+PI we\textquotesingle{}re using in case this changes in the future
\item {\bfseries sensor\+\_\+id}\+: A unique sensor identifier that is used to differentiate this specific sensor instance from any others that are present on the system or in the sensor network (must match the sensor\+\_\+id value in the corresponding \hyperlink{_adafruit___sensor_8h_structsensor__t}{sensor\+\_\+t} enum above!)
\item {\bfseries type}\+: the sensor type, based on {\bfseries sensors\+\_\+type\+\_\+t} in sensors.\+h
\item {\bfseries timestamp}\+: time in milliseconds when the sensor value was read
\item {\bfseries data\mbox{[}4\mbox{]}}\+: An array of four 32-\/bit values that allows us to encapsulate any type of sensor data via a simple union (further described below)
\end{DoxyItemize}

{\bfseries Required Functions}

In addition to the two standard types and the sensor type enum, all drivers based on \hyperlink{class_adafruit___sensor}{Adafruit\+\_\+\+Sensor} must also implement the following two functions\+:


\begin{DoxyCode}
1 bool getEvent(sensors\_event\_t*);
\end{DoxyCode}
 Calling this function will populate the supplied \hyperlink{_adafruit___sensor_8h_structsensors__event__t}{sensors\+\_\+event\+\_\+t} reference with the latest available sensor data. You should call this function as often as you want to update your data.


\begin{DoxyCode}
1 void getSensor(sensor\_t*);
\end{DoxyCode}
 Calling this function will provide some basic information about the sensor (the sensor name, driver version, min and max values, etc.

{\bfseries Standardised SI values for \hyperlink{_adafruit___sensor_8h_structsensors__event__t}{sensors\+\_\+event\+\_\+t}}

A key part of the abstraction layer is the standardisation of values on SI units of a particular scale, which is accomplished via the data\mbox{[}4\mbox{]} union in \hyperlink{_adafruit___sensor_8h_structsensors__event__t}{sensors\+\_\+event\+\_\+t} above. This 16 byte union includes fields for each main sensor type, and uses the following SI units and scales\+:


\begin{DoxyItemize}
\item {\bfseries acceleration}\+: values are in {\bfseries meter per second per second} (m/s$^\wedge$2)
\item {\bfseries magnetic}\+: values are in {\bfseries micro-\/\+Tesla} (uT)
\item {\bfseries orientation}\+: values are in {\bfseries degrees}
\item {\bfseries gyro}\+: values are in {\bfseries rad/s}
\item {\bfseries temperature}\+: values in {\bfseries degrees centigrade} (Celsius)
\item {\bfseries distance}\+: values are in {\bfseries centimeters}
\item {\bfseries light}\+: values are in {\bfseries SI lux} units
\item {\bfseries pressure}\+: values are in {\bfseries hectopascal} (h\+Pa)
\item {\bfseries relative\+\_\+humidity}\+: values are in {\bfseries percent}
\item {\bfseries current}\+: values are in {\bfseries milliamps} (mA)
\item {\bfseries voltage}\+: values are in {\bfseries volts} (V)
\item {\bfseries color}\+: values are in 0..1.\+0 R\+GB channel luminosity and 32-\/bit R\+G\+BA format
\end{DoxyItemize}

\subsection*{The Unified Driver Abstraction Layer in Practice}

Using the unified sensor abstraction layer is relatively easy once a compliant driver has been created.

Every compliant sensor can now be read using a single, well-\/known \textquotesingle{}type\textquotesingle{} (\hyperlink{_adafruit___sensor_8h_structsensors__event__t}{sensors\+\_\+event\+\_\+t}), and there is a standardised way of interrogating a sensor about its specific capabilities (via \hyperlink{_adafruit___sensor_8h_structsensor__t}{sensor\+\_\+t}).

An example of reading the \href{https://github.com/adafruit/Adafruit_TSL2561}{\tt T\+S\+L2561} light sensor can be seen below\+:


\begin{DoxyCode}
1 Adafruit\_TSL2561 tsl = Adafruit\_TSL2561(TSL2561\_ADDR\_FLOAT, 12345);
2 ...
3 /* Get a new sensor event */ 
4 sensors\_event\_t event;
5 tsl.getEvent(&event);
6 
7 /* Display the results (light is measured in lux) */
8 if (event.light)
9 \{
10   Serial.print(event.light); Serial.println(" lux");
11 \}
12 else
13 \{
14   /* If event.light = 0 lux the sensor is probably saturated
15      and no reliable data could be generated! */
16   Serial.println("Sensor overload");
17 \}
\end{DoxyCode}


Similarly, we can get the basic technical capabilities of this sensor with the following code\+:


\begin{DoxyCode}
1 sensor\_t sensor;
2 
3 sensor\_t sensor;
4 tsl.getSensor(&sensor);
5 
6 /* Display the sensor details */
7 Serial.println("------------------------------------");
8 Serial.print  ("Sensor:       "); Serial.println(sensor.name);
9 Serial.print  ("Driver Ver:   "); Serial.println(sensor.version);
10 Serial.print  ("Unique ID:    "); Serial.println(sensor.sensor\_id);
11 Serial.print  ("Max Value:    "); Serial.print(sensor.max\_value); Serial.println(" lux");
12 Serial.print  ("Min Value:    "); Serial.print(sensor.min\_value); Serial.println(" lux");
13 Serial.print  ("Resolution:   "); Serial.print(sensor.resolution); Serial.println(" lux");  
14 Serial.println("------------------------------------");
15 Serial.println("");
\end{DoxyCode}
 